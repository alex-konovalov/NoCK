% generated by GAPDoc2LaTeX from XML source (Frank Luebeck)
\documentclass[a4paper,11pt]{report}

\usepackage[top=37mm,bottom=37mm,left=27mm,right=27mm]{geometry}
\sloppy
\pagestyle{myheadings}
\usepackage{amssymb}
\usepackage[latin1]{inputenc}
\usepackage{makeidx}
\makeindex
\usepackage{color}
\definecolor{FireBrick}{rgb}{0.5812,0.0074,0.0083}
\definecolor{RoyalBlue}{rgb}{0.0236,0.0894,0.6179}
\definecolor{RoyalGreen}{rgb}{0.0236,0.6179,0.0894}
\definecolor{RoyalRed}{rgb}{0.6179,0.0236,0.0894}
\definecolor{LightBlue}{rgb}{0.8544,0.9511,1.0000}
\definecolor{Black}{rgb}{0.0,0.0,0.0}

\definecolor{linkColor}{rgb}{0.0,0.0,0.554}
\definecolor{citeColor}{rgb}{0.0,0.0,0.554}
\definecolor{fileColor}{rgb}{0.0,0.0,0.554}
\definecolor{urlColor}{rgb}{0.0,0.0,0.554}
\definecolor{promptColor}{rgb}{0.0,0.0,0.589}
\definecolor{brkpromptColor}{rgb}{0.589,0.0,0.0}
\definecolor{gapinputColor}{rgb}{0.589,0.0,0.0}
\definecolor{gapoutputColor}{rgb}{0.0,0.0,0.0}

%%  for a long time these were red and blue by default,
%%  now black, but keep variables to overwrite
\definecolor{FuncColor}{rgb}{0.0,0.0,0.0}
%% strange name because of pdflatex bug:
\definecolor{Chapter }{rgb}{0.0,0.0,0.0}
\definecolor{DarkOlive}{rgb}{0.1047,0.2412,0.0064}


\usepackage{fancyvrb}

\usepackage{mathptmx,helvet}
\usepackage[T1]{fontenc}
\usepackage{textcomp}


\usepackage[
            pdftex=true,
            bookmarks=true,        
            a4paper=true,
            pdftitle={Written with GAPDoc},
            pdfcreator={LaTeX with hyperref package / GAPDoc},
            colorlinks=true,
            backref=page,
            breaklinks=true,
            linkcolor=linkColor,
            citecolor=citeColor,
            filecolor=fileColor,
            urlcolor=urlColor,
            pdfpagemode={UseNone}, 
           ]{hyperref}

\newcommand{\maintitlesize}{\fontsize{50}{55}\selectfont}

% write page numbers to a .pnr log file for online help
\newwrite\pagenrlog
\immediate\openout\pagenrlog =\jobname.pnr
\immediate\write\pagenrlog{PAGENRS := [}
\newcommand{\logpage}[1]{\protect\write\pagenrlog{#1, \thepage,}}
%% were never documented, give conflicts with some additional packages

\newcommand{\GAP}{\textsf{GAP}}

%% nicer description environments, allows long labels
\usepackage{enumitem}
\setdescription{style=nextline}

%% depth of toc
\setcounter{tocdepth}{1}





%% command for ColorPrompt style examples
\newcommand{\gapprompt}[1]{\color{promptColor}{\bfseries #1}}
\newcommand{\gapbrkprompt}[1]{\color{brkpromptColor}{\bfseries #1}}
\newcommand{\gapinput}[1]{\color{gapinputColor}{#1}}


\begin{document}

\logpage{[ 0, 0, 0 ]}
\begin{titlepage}
\mbox{}\vfill

\begin{center}{\maintitlesize \textbf{NoCK\mbox{}}}\\
\vfill

\hypersetup{pdftitle=NoCK}
\markright{\scriptsize \mbox{}\hfill NoCK \hfill\mbox{}}
{\Huge \textbf{Computing obstruction for the existence of compact Clifford-Klein form\mbox{}}}\\
\vfill

{\Huge Version 1.3\mbox{}}\\[1cm]
{July 2019 \mbox{}}\\[1cm]
\mbox{}\\[2cm]
{\Large \textbf{ Maciej Boche{\a'n}ski   \mbox{}}}\\
{\Large \textbf{ Piotr Jastrz{\k e}bski    \mbox{}}}\\
{\Large \textbf{ Anna Szczepkowska   \mbox{}}}\\
{\Large \textbf{ Aleksy Tralle    \mbox{}}}\\
{\Large \textbf{ Artur Woike    \mbox{}}}\\
\hypersetup{pdfauthor= Maciej Boche{\a'n}ski   ;  Piotr Jastrz{\k e}bski    ;  Anna Szczepkowska   ;  Aleksy Tralle    ;  Artur Woike    }
\end{center}\vfill

\mbox{}\\
{\mbox{}\\
\small \noindent \textbf{ Maciej Boche{\a'n}ski   }  Email: \href{mailto://mabo@matman.uwm.edu.pl} {\texttt{mabo@matman.uwm.edu.pl}}\\
  Address: \begin{minipage}[t]{8cm}\noindent
 Faculty of Mathematics and Computer Science,\\
 University of Warmia and Mazury in Olsztyn\\
 Sloneczna 54 Street, \\
 10-710 Olsztyn, Poland \end{minipage}
}\\
{\mbox{}\\
\small \noindent \textbf{ Piotr Jastrz{\k e}bski    }  Email: \href{mailto://piojas@matman.uwm.edu.pl} {\texttt{piojas@matman.uwm.edu.pl}}\\
  Homepage: \href{http://wmii.uwm.edu.pl/~piojas/} {\texttt{http://wmii.uwm.edu.pl/\texttt{\symbol{126}}piojas/}}\\
  Address: \begin{minipage}[t]{8cm}\noindent
 Faculty of Mathematics and Computer Science,\\
 University of Warmia and Mazury in Olsztyn\\
 Sloneczna 54 Street, \\
 10-710 Olsztyn, Poland \end{minipage}
}\\
{\mbox{}\\
\small \noindent \textbf{ Anna Szczepkowska   }  Email: \href{mailto://anna.szczepkowska@matman.uwm.edu.pl} {\texttt{anna.szczepkowska@matman.uwm.edu.pl}}\\
  Address: \begin{minipage}[t]{8cm}\noindent
 Faculty of Mathematics and Computer Science,\\
 University of Warmia and Mazury in Olsztyn\\
 Sloneczna 54 Street, \\
 10-710 Olsztyn, Poland \end{minipage}
}\\
{\mbox{}\\
\small \noindent \textbf{ Aleksy Tralle    }  Email: \href{mailto://tralle@matman.uwm.edu.pl} {\texttt{tralle@matman.uwm.edu.pl}}\\
  Homepage: \href{http://wmii.uwm.edu.pl/~tralle/} {\texttt{http://wmii.uwm.edu.pl/\texttt{\symbol{126}}tralle/}}\\
  Address: \begin{minipage}[t]{8cm}\noindent
 Faculty of Mathematics and Computer Science,\\
 University of Warmia and Mazury in Olsztyn\\
 Sloneczna 54 Street, \\
 10-710 Olsztyn, Poland \end{minipage}
}\\
{\mbox{}\\
\small \noindent \textbf{ Artur Woike    }  Email: \href{mailto://awoike@matman.uwm.edu.pl} {\texttt{awoike@matman.uwm.edu.pl}}\\
  Homepage: \href{http://wmii.uwm.edu.pl/~awoike/} {\texttt{http://wmii.uwm.edu.pl/\texttt{\symbol{126}}awoike/}}\\
  Address: \begin{minipage}[t]{8cm}\noindent
 Faculty of Mathematics and Computer Science,\\
 University of Warmia and Mazury in Olsztyn\\
 Sloneczna 54 Street, \\
 10-710 Olsztyn, Poland \end{minipage}
}\\
\end{titlepage}

\newpage\setcounter{page}{2}
{\small 
\section*{Abstract}
\logpage{[ 0, 0, 1 ]}
 In this package we develop functions for an algorithm designed to find
homogeneous spaces of semisimple non-compact Lie groups which do not admit
compact Clifford-Klein forms. \mbox{}}\\[1cm]
{\small 
\section*{Copyright}
\logpage{[ 0, 0, 3 ]}
 NoCK Package is free software; you can redistribute it and/or modify it under
the terms of the \href{http://www.fsf.org/licenses/gpl.html} {GNU General Public License} as published by the Free Software Foundation; either version 2 of the License,
or (at your option) any later version. \mbox{}}\\[1cm]
{\small 
\section*{Acknowledgements}
\logpage{[ 0, 0, 2 ]}
 We thank Willem de Graaf for his help in getting some literature sources. \mbox{}}\\[1cm]
\newpage

\def\contentsname{Contents\logpage{[ 0, 0, 4 ]}}

\tableofcontents
\newpage

 
\chapter{\textcolor{Chapter }{Notation}}\logpage{[ 1, 0, 0 ]}
\hyperdef{L}{X7DD31B407B9402A3}{}
{
  We use the notation and convention for real Lie algebras as is from CoReLG
Package, \cite{CoReLG}. 
\begin{Verbatim}[commandchars=!@|,fontsize=\small,frame=single,label=Example]
  !gapprompt@gap>| !gapinput@G:=RealFormById( "E", 7,3);|
  <Lie algebra of dimension 133 over SqrtField>
  !gapprompt@gap>| !gapinput@rankG:=Dimension(CartanSubalgebra(G));|
  7
  !gapprompt@gap>| !gapinput@rankRG:=Dimension(CartanSubspace(G));|
  3
  !gapprompt@gap>| !gapinput@dimG:=Dimension(G);|
  133
  !gapprompt@gap>| !gapinput@P:=CartanDecomposition( G ).P;|
  <vector space over SqrtField, with 54 generators>
  !gapprompt@gap>| !gapinput@dimPforG:=Dimension(P);|
  54
  !gapprompt@gap>| !gapinput@K:=CartanDecomposition( G ).K;|
  <Lie algebra of dimension 79 over SqrtField>
  !gapprompt@gap>| !gapinput@rankK:= Dimension(CartanSubalgebra(K));|
  7
  !gapprompt@gap>| !gapinput@dimK:= Dimension(K);|
  79
\end{Verbatim}
 Classification can be found in Table 9 in \cite{onvin}, p. 312-317. }

 
\chapter{\textcolor{Chapter }{Obstruction for the existence of compact Clifford-Klein form}}\logpage{[ 2, 0, 0 ]}
\hyperdef{L}{X7DD2840E797415E3}{}
{
  In this chapter we describe functions for algorithm from \cite{our}. 
\section{\textcolor{Chapter }{Technical functions}}\logpage{[ 2, 1, 0 ]}
\hyperdef{L}{X805C42557996F58A}{}
{
  

\subsection{\textcolor{Chapter }{NonCompactDimension}}
\logpage{[ 2, 1, 1 ]}\nobreak
\hyperdef{L}{X832D9E887973AABE}{}
{\noindent\textcolor{FuncColor}{$\triangleright$\enspace\texttt{NonCompactDimension({\mdseries\slshape G})\index{NonCompactDimension@\texttt{NonCompactDimension}}
\label{NonCompactDimension}
}\hfill{\scriptsize (function)}}\\


 For a real Lie algebra $G$ constructed by the function \mbox{\texttt{\mdseries\slshape RealFormById}} (from \cite{CoReLG}), this function returns the non-compact dimension of $G$ (dimension of a non-compact part in Cartan decomposition of $G$). 
\begin{Verbatim}[commandchars=!@|,fontsize=\small,frame=single,label=Example]
  !gapprompt@gap>| !gapinput@G:=RealFormById("E",6,2); # E6(6)|
  <Lie algebra of dimension 78 over SqrtField>
  !gapprompt@gap>| !gapinput@dG:=NonCompactDimension(G);|
  42
\end{Verbatim}
 }



\subsection{\textcolor{Chapter }{PCoefficients}}
\logpage{[ 2, 1, 2 ]}\nobreak
\hyperdef{L}{X868FC1B77B4B4D4C}{}
{\noindent\textcolor{FuncColor}{$\triangleright$\enspace\texttt{PCoefficients({\mdseries\slshape type, rank})\index{PCoefficients@\texttt{PCoefficients}}
\label{PCoefficients}
}\hfill{\scriptsize (function)}}\\


 Let $G$ be a compact connected Lie group of the type \mbox{\texttt{\mdseries\slshape type}} and the rank \mbox{\texttt{\mdseries\slshape rank}}. Let $\Lambda\,P_{G}=\Lambda (y_1,...,y_l)$ be the exterior algebra over the spaces $P_G$ of the primitive elements in $H^*(G)$. Denote the degrees as follows $|y_j|=2p_j-1,j=1,...,l$. This function returns coefficients $p_1,\ldots,p_l$. 
\begin{Verbatim}[commandchars=!@|,fontsize=\small,frame=single,label=Example]
  !gapprompt@gap>| !gapinput@PCoefficients("D",5);|
  [ 2, 4, 6, 8, 5 ]
\end{Verbatim}
 }



\subsection{\textcolor{Chapter }{PCalculate}}
\logpage{[ 2, 1, 3 ]}\nobreak
\hyperdef{L}{X827DC41787C8BC7B}{}
{\noindent\textcolor{FuncColor}{$\triangleright$\enspace\texttt{PCalculate({\mdseries\slshape pi, qi})\index{PCalculate@\texttt{PCalculate}}
\label{PCalculate}
}\hfill{\scriptsize (function)}}\\


 Here $pi=\{ p_1,\ldots,p_l\}$ and $qi=\{ q_1,\ldots,q_m\}$ are sets of coefficients ($l\geq m$). This function returns the polynomial: $P(t)=\prod_{j=m+1}^l(1+t^{2p_j-1})\prod_{i=1}^m(1-t^{2p_i})/(1-t^{2q_i})$. 
\begin{Verbatim}[commandchars=!@|,fontsize=\small,frame=single,label=Example]
  !gapprompt@gap>| !gapinput@PCalculate([4,2,3],[2,2]);   |
  t^9+t^5+t^4+1
\end{Verbatim}
 }



\subsection{\textcolor{Chapter }{AllZeroDH}}
\logpage{[ 2, 1, 4 ]}\nobreak
\hyperdef{L}{X87164CA87A56E5B8}{}
{\noindent\textcolor{FuncColor}{$\triangleright$\enspace\texttt{AllZeroDH({\mdseries\slshape type, rank, id})\index{AllZeroDH@\texttt{AllZeroDH}}
\label{AllZeroDH}
}\hfill{\scriptsize (function)}}\\


 Let $G^C$ be a complex Lie algebra of the type \mbox{\texttt{\mdseries\slshape type}} and the rank \mbox{\texttt{\mdseries\slshape rank}}. Let $G$ be a real form of $G^C$ with the index \mbox{\texttt{\mdseries\slshape id}} (see \mbox{\texttt{\mdseries\slshape RealFormsInformation}},\cite{CoReLG}). This function returns the set of degrees of $P(t)$ that have zero coefficients over all permutation (see Section 7 in \cite{our}). 
\begin{Verbatim}[commandchars=!@|,fontsize=\small,frame=single,label=Example]
  !gapprompt@gap>| !gapinput@AllZeroDH("F",4,2); |
  [ 1, 2, 3, 5, 6, 7, 9, 10, 11, 13, 14, 15, 17, 18, 19, 21, 22, 23, 25, 26, 27 ]
\end{Verbatim}
 }

 }

 }

 
\chapter{\textcolor{Chapter }{Algorithm example}}\logpage{[ 3, 0, 0 ]}
\hyperdef{L}{X7F57F15D7A4099A1}{}
{
  In this chapter we use additionaly functions from the following packages:
CoReLG \cite{CoReLG} and SLA \cite{SLA}. We will show in detail the split case (for a non-split case you should use
algoritm to generate regular subalgebras from \cite{DFG}). For example, we take $G=\mathfrak{e}_{6(6)}$ (tuple "E",6,2 in CoReLG notation). We calculate \mbox{\texttt{\mdseries\slshape AllZeroDH}} on it. 
\begin{Verbatim}[commandchars=!@|,fontsize=\small,frame=single,label=Example]
  !gapprompt@gap>| !gapinput@AllZeroDH("E",6,2);|
  [ 1, 2, 3, 4, 5, 6, 7, 10, 11, 12, 13, 14, 15, 18, 19, 20, 21, 22, 23, 24, 27, 
   28, 29, 30, 31, 32, 35, 36, 37, 38, 39, 40, 41 ]
\end{Verbatim}
 We generate all regular subalgebras of complexification. 
\begin{Verbatim}[commandchars=!@|,fontsize=\small,frame=single,label=Example]
  !gapprompt@gap>| !gapinput@GC:=SimpleLieAlgebra("E",6,Rationals);;  |
  !gapprompt@gap>| !gapinput@REG:=RegularSemisimpleSubalgebras(GC);;|
  !gapprompt@gap>| !gapinput@L0:=List( REG, SemiSimpleType );   |
  [ "A1", "A1 A1", "A2 A1", "A4", "D5", "A4 A1", "A2 A1 A1", "A2 A1 A2", "A3 A1", 
   "A1 A1 A1", "A2", "A3", "A5", "A2 A2", "D4", "A5 A1", "A3 A1 A1", "A1 A1 A1 A1", 
   "A2 A2 A2" ]
\end{Verbatim}
 For each subalgebras we take the split real form and calculate its non-compact
dimension. 
\begin{Verbatim}[commandchars=!@|,fontsize=\small,frame=single,label=Example]
  !gapprompt@gap>| !gapinput@L0[4]; |
  "A4"
  !gapprompt@gap>| !gapinput@RealFormsInformation( "A", 4 ); |
  
    There are 4 simple real forms with complexification A4
      1 is of type su(5), compact form
      2 - 3 are of type su(p,5-p) with 1 <= p <= 2
      4 is of type sl(5,R)
    Index '0' returns the realification of A4
  
  !gapprompt@gap>| !gapinput@G:=RealFormById("A",4,4);;     |
  !gapprompt@gap>| !gapinput@NonCompactDimension( G );      |
  14
\end{Verbatim}
 Number 14 is in output of \mbox{\texttt{\mdseries\slshape AllZeroDH}} function, so for $\mathfrak{g}=e_{6(6)}$ and $\mathfrak{h}=\mathfrak{sl}(5,\mathbb{R})$ corresponding homogeneous spaces $G/H$ do not have compact Clifford{\textendash}Klein forms. 
\begin{Verbatim}[commandchars=!@|,fontsize=\small,frame=single,label=Example]
  !gapprompt@gap>| !gapinput@L0[5];                                                          |
  "D5"
  !gapprompt@gap>| !gapinput@RealFormsInformation( "D", 5 ); |
  
    There are 7 simple real forms with complexification D5
      1 is of type so(10), compact form
      2 - 3 are of type so(2p,10-2p) with 1 <= p <= 2
      4 is of type so*(10)
      5 is of type so(9,1)
      6 - 7 are of type so(2p+1,10-2p-1) with 1 <= p <= 2
    Index '0' returns the realification of D5
  
  !gapprompt@gap>| !gapinput@G:=RealFormById("D",5,7);; |
  !gapprompt@gap>| !gapinput@NonCompactDimension( G );                                       |
  25
\end{Verbatim}
 Number 25 is not in output of \mbox{\texttt{\mdseries\slshape AllZeroDH}} function, so for $\mathfrak{g}=e_{6(6)}$ and $\mathfrak{h}=\mathfrak{so}(5,5)$ our algoritm does not provide a solution to the problem. }

 \def\bibname{References\logpage{[ "Bib", 0, 0 ]}
\hyperdef{L}{X7A6F98FD85F02BFE}{}
}

\bibliographystyle{alpha}
\bibliography{NoCKbib.xml}

\addcontentsline{toc}{chapter}{References}

\def\indexname{Index\logpage{[ "Ind", 0, 0 ]}
\hyperdef{L}{X83A0356F839C696F}{}
}

\cleardoublepage
\phantomsection
\addcontentsline{toc}{chapter}{Index}


\printindex

\newpage
\immediate\write\pagenrlog{["End"], \arabic{page}];}
\immediate\closeout\pagenrlog
\end{document}
